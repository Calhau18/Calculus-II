\documentclass{article}
\title{Resumos de \\ Cálculo Diferencial e Integral II}
\date{2020}
\author{João Rocha}

\usepackage[utf8]{inputenc}
\usepackage{indentfirst}
\usepackage{comment}
\usepackage{amsmath}
\usepackage{amssymb}
\usepackage{mathtools, nccmath}
\usepackage{geometry}
\usepackage{graphicx}
 \geometry{
 a4paper,
 total={129mm,170mm},
 left=23mm,
 top=23mm,
 bottom=28mm,
 right=28mm,
 }

\begin{document}
\maketitle

\vspace*{\fill}
\textbf{AVISO}:\\ 
Este texto é essencialmente um resumo dos conhecimentos mais essenciais da cadeira de Cálculo II, não tendo portanto o objetivo de entrar por explicações em detalhe dos conceitos abordados. Desse modo, o objetivo principal deste documento é o de consulta, e não o de esclarecimento. No entanto, alguns temas são acompanhados de explicações breves, de forma a relembrar a intuição por trás dos mesmos. 
\newpage

\section{O espaço $\mathbb{R}^n$}
\subsection{Funções em $\mathbb{R}^n$}
A cadeira de Cálculo I centra-se no estudo de funções \textbf{reais de variável real}, isto é funções que fazem correspondências dos reais para os reais.\\
No entanto, isto revela-se frequentemente insuficente. Cálculo é frequentemente útil para analisar certos sistemas, e estes sistemas podem ter por vezes mais do que um variável "em jogo".\\
A cadeira de Cálculo II tenta então generalizar o estudo da função para situações em que há mais do que uma variável.\\
Definimos então o conjunto $\mathbb{R}^n$ como o conjunto dos pontos $x=(x_1, x_2, \cdots, x_n)$, com $x_1, x_2, \cdots , x_n \in \mathbb{R}$, tal como seria definido numa cadeira de Álgebra Linear. Como é de assumir que qualquer aluno a estudar cálculo com várias variáveis está familiarizado com o conceito de $\mathbb{R}^n$ não vamos entrar em mais detalhe/formalidades e assumir que $\mathbb{R}^n$ tem as propriedades axiomáticas como definidas numa cadeira de Álgebra Linear.\\
O conceito de $\mathbb{R}^n$ permite-nos extender o conceito de função. Uma \textbf{função vetorial de variável vetorial} é qualquer transformação que faz corresponder a cada elemento de um conjunto $D \subset \mathbb{R}^n$ um e um só elemento de $\mathbb{R}^m$ ($m,n \in \mathbb{N}$) (note-se que em Álgebra Linear já se falou de funções - ou transformações - de $\mathbb{R}^n$ em $\mathbb{R}^m$, mas em Cálculo II não nos vamos restringir a funções/transformações lineares).\\
Observe-se que muito do estudo feito sobre funções só é possível devido ao conceito de limite: a ideia de algo a \textbf{"aproximar-se"} muito de outro algo. Para trabalhar matemáticamente com esta ideia de proximidade, é preciso definir \textbf{distância}. Relembramo-nos então dos seguintes conceitos de AL:
\begin{itemize}
	\item O \text{módulo} de um ponto (ou vetor) em $\mathbb{R}^n$ é dado (usualmente) por $||x|| = \sqrt{x_1^2 + x_2^2 + \cdots + x_n^2}$ em que $x_1, \cdots , x_n$ são as coordenadas de $x$.
	\item A \textbf{distância} entre $x, y \in \mathbb{R}^n$ é dada por $||x-y||$.
\end{itemize}

\subsection{Topologia de $\mathbb{R}^n$}
A noção de distância em $\mathbb{R}^n$ permite-nos extender alguns dos conceitos topológicos dos reais para $\mathbb{R}^n$.\\
Começamos com um dos conceitos mais fundamentais: o de vizinhança. Definimos uma \textbf{bola} de raio $\epsilon \in \mathbb{R}^+$ e centro $a \in \mathbb{R}^n$ como o conjunto $B_\epsilon(a) = \{ x \in \mathbb{R}^n : ||x-a|| < \epsilon \}$. É evidente que o conceito de bola é uma extensão do conceito de vizinhançaem $\mathbb{R}^n$, pelo que terá um papel muito semelhante a essa estrutura.\\
Com este conceito, podemos então definir conjuntos de forma semelhande ao feito em CDI1:
\begin{itemize}
\item $a$ é um \textbf{ponto interior} de $A$ se existe $r$ tal que $B_r(a) \subset A$ \vspace{-0.8mm}
\item $a$ é um \textbf{ponto exterior} de $A$ se existe $r$ tal que $B_r(a) \subset \mathbb{R}^n/A$ \vspace{-0.8mm}
\item $a$ é um \textbf{ponto fronteiriço} de $A$ se não é interior nem exterior a $A$ \vspace{-0.8mm}
\item $a$ é um \textbf{ponto aderente} a $A$ se para qualquer $r$, $B_r(a) \cap A \neq \emptyset$ \vspace{-0.8mm}
\end{itemize}
e chamamos: \vspace{-0.8mm}
\begin{itemize}
\item \textbf{Interior} de $A$ ($int \, A$) ao conjunto de pontos interiores de A \vspace{-0.8mm}
\item \textbf{Exterior} de $A$ ($ext \, A$) ao conjunto de pontos exteriores de A \vspace{-0.8mm}
\item \textbf{Fronteira} de $A$ ao ($\partial A$) conjunto de pontos fronteirições de A \vspace{-0.8mm}
\item \textbf{Fecho} ou \textbf{aderência} de $A$ ($\overline{A}$) ao conjunto de pontos aderentes de A ($\overline{A}=int \, A \cup \partial A$) \vspace{-0.8mm}
\end{itemize}
Observe-se que para qualquer $A \subset \mathbb{R}^n$, temos que 
$int \, A \cup \partial A \cup ext \, A = \mathbb{R}^n$ e $int \, A \cap \partial A \cap ext \, A = \emptyset$.\\
Um conjunto $A \in \mathbb{R}$ diz-se: \vspace{-0.8mm}
\begin{itemize}
\item \textbf{aberto} se $A= int \, A$ \vspace{-1mm}
\item \textbf{fechado} se $A= \overline{A}$ ($\mathbb{R}/A$ é aberto ou $\partial A \subset A$) \vspace{-0.8mm}
\item Um conjunto $A$ diz-se \textbf{limitado} se houver $x \in \mathbb{R}$ e $R \in \mathbb{R}^+$ tal que $A \subset B_R(x)$.
\item \textbf{compacto} se é fechado e limitado \vspace{-0.8mm}
\end{itemize}
A interseção/união de uma família de conjuntos abertos/fechados é também aberto/fechado.

\subsection{Sucessões}
Recorda-se que uma sucessão é essencialmente uma bijeção/correspondência entre um conjunto $\{ u_n: n \in \mathbb{N}^+\}$ e $\mathbb{N}^+$.
Seja $u_n$ uma sucessão de termos em $\mathbb{R}^n$. Observa-se que cada coordenada de $u_n$ descreve ela própria um sucessão de termos em $\mathbb{R}^n$. Assim, para cada $j \leq m$ damos o nome de \textbf{sucessões coordenadas} às sucessões $(u_{j_n})$ cujo termo de ordem $k$ corresponde à $j$-ésima coordenada de $u_k$.\\
As opções básicas das sucessões em $\mathbb{R}^n$ definem-se de forma óbvia tendo em conta a soma vetorial e das sucessões coordenadas em $\mathbb{R}$.\\
Diz-se que uma sucessão $(u_n) \subset \mathbb{R}^n$ tende/converge para $u \in \mathbb{R}^n$ (e escreve-se $u_n \to u$) se e só se, para qualquer $\epsilon \in \mathbb{R}^+$ existe $p \in \mathbb{N}^+$ tal que $u_n \in B_\epsilon(u)$ para todo o $n>p$. Analogamente, vemos que esta definição é equivalente a verificar se a sucessão $d_n = || u_n - u ||$ é infinitesimal. Uma sucessão que convirja para $u \in \mathbb{R}^n$ diz-se convergente.\\
Note-se que as definições a cima enunciadas de convergência são pouco práticas. De facto, é bastante mais prático usar o seguinte facto:\\
Uma sucessão $(u_n)$ converge para $a = (a_1, a_2, \cdots , a_n) \in \mathbb{R}^n$ se e só se as sucessões coordenadas $u_{j_n}$ tendem para $a_j$ para todo o $j \in \{ 1,2, \cdots , n \}$. É então fácil verificar que o limite de uma sucessão em $\mathbb{R}^n$ é único (se existir).\\
Observe-se que qualquer sucessão de termos em $X \in \mathbb{R}^n$ tem limite na aderência de $X$.\\
Agora, mais uma vez, é fácil determinar os resultados esperados sobre operações com limites, tendo em conta a forma como estão definidas a convergência em $\mathbb{R}^n$ e as operações sobre limites em $\mathbb{R}$.\\
Diz-se ainda que uma sucessão é limitada se e só se o conjunto dos seus termos também o for. Tendo em conta os resultados a cima, é fácil concluir que uma sucessão é limitada se e só se as suas sucessões coordenadas também o forem. Verificam-se as seguintes propriedades:
\begin{itemize}
	\item Qualquer sucessão convergente é limitada.
	\item \textbf{Teorema de Bolzano-Weiestrass}: Qualquer sucessão limitada tem sub-sucessões convergentes.
\end{itemize}

\subsection{Continuidade}
Em Cálculo I introduzimos continuidade de acordo com a seguinte ideia: uma função $f$ é contínua num ponto $a$ se conseguirmos que todos os pontos de $f$ estão tão perto quanto quisermos de $f(a)$, se considerarmos apenas pontos suficientemente perto de $a$. Tendo em conta a noção de distância em $\mathbb{R}^n$, definimos continuidade da seguinte forma:\\
Uma função $f: D \subset \mathbb{R}^n \to \mathbb{R}^m$ diz-se contínua num ponto $a \in \mathbb{R}^n$ se para qualquer bola de raio $\delta \in \mathbb{R}^+$ centrada em $f(a)$, existe uma bola de raio $\epsilon \in \mathbb{R}^+$ centrada em $a$ tal que $x \in B_\epsilon(a) \, \cap \, D \Rightarrow f(x) \in B_\delta(f(a))$ ($\forall \delta \in \mathbb{R}^+, \, \exists \epsilon \in \mathbb{R}^+ : ||x-a|| < \epsilon \vee x \in D \Rightarrow ||f(x)-f(a)|| < \delta$).\\
Tal como em $\mathbb{R}$ (definição de Heine), podemos definir continuidade através de sucessões. Uma funçao $f : D \subset \mathbb{R}^n \to \mathbb{R}^m$ é contínua em $a \in D$ se e só se para qualquer sucessão $(u_n)$ de termos em $D$ convergente para $a$ se tem que $f(u_k) \to f(a)$.\\
Mais uma vez, podemos obter resultados semelhantes aos elaborados em CDI-I para as operações sobre funções contínuas e continuidade das funções resultantes.\\
Tal como nas sucessões, por vezes o estudo da continuidade de uma função torna-se mais simples se considerarmos as \textbf{funções coordenadas}. Seja uma função $f: D \subset \mathbb{R}^n \to \mathbb{R}^m$. Se considerarmos apenas o que $f$ faz à $k$-ésima coordenada dos objetos de $D$, obtemos uma função escalar ($f_k(x)$, a qual designamos de $k$-ésima função coordenada de $f$). Tal função $f$ será contínua num ponto $a \in D$ se e só se as $m$ funções coordenadas também o forem.\\

\subsection{Limites}


\end{document}
