\documentclass{article}
\title{Resumos de \\ Cálculo Diferencial e Integral II}
\date{2020}
\author{João Rocha}

\usepackage[utf8]{inputenc}
\usepackage{indentfirst}
\usepackage{comment}
\usepackage{amsmath}
\usepackage{amssymb}
\usepackage{mathtools, nccmath}
\usepackage{geometry}
\usepackage{graphicx}
 \geometry{
 a4paper,
 total={129mm,170mm},
 left=23mm,
 top=23mm,
 bottom=28mm,
 right=28mm,
 }

\begin{document}
\maketitle

\vspace*{\fill}
\textbf{AVISO}:\\ 
Este texto é essencialmente um resumo dos conhecimentos mais essenciais da cadeira de Cálculo II, não tendo portanto o objetivo de entrar por explicações em detalhe dos conceitos abordados. Desse modo, o objetivo principal deste documento é o de consulta, e não o de esclarecimento. No entanto, alguns temas são acompanhados de explicações breves, de forma a relembrar a intuição por trás dos mesmos. 
\newpage

\section{O espaço $\mathbb{R}^n$}
\subsection{Funções em $\mathbb{R}^n$}
A cadeira de Cálculo I centra-se no estudo de funções \textbf{reais de variável real}, isto é funções que fazem correspondências dos reais para os reais.\\
No entanto, isto revela-se frequentemente insuficente. Cálculo é frequentemente útil para analisar certos sistemas, e estes sistemas podem ter por vezes mais do que um variável "em jogo".\\
A cadeira de Cálculo II tenta então generalizar o estudo da função para situações em que há mais do que uma variável em consideração.\\
Definimos então o conjunto $\mathbb{R}^n$ como o conjunto dos pontos $x=(x_1, x_2, \cdots, x_n)$, com $x_1, x_2, \cdots , x_n \in \mathbb{R}$, tal como seria definido numa cadeira de Álgebra Linear. Como é de assumir que qualquer aluno a estudar cálculo com várias variáveis está familiarizado com o conceito de $\mathbb{R}^n$ não vamos entrar em mais detalhe/formalidades e assumir que $\mathbb{R}^n$ tem as propriedades axiomáticas como definidas numa cadeira de Álgebra Linear.\\
Definimos uma \textbf{função vetorial de variável vetorial} como qualquer transformação que faz corresponder a cada elemento de um conjunto $D \subset \mathbb{R}^n$ um e um só elemento de $\mathbb{R}^m$ ($m,n \in \mathbb{N}$).\\
Para realizar o estude de funções vetoriais, é extremamente importante a noção de limite. Como esta noção depende da noção de distância, começamos por definir este conceito:
\begin{itemize}
	\item O \text{módulo} de um ponto (ou vetor) em $\mathbb{R}^n$ é dado (usualmente) por $||x|| = \sqrt{x_1^2 + x_2^2 + \cdots + x_n^2}$ em que $x_1, \cdots , x_n$ são as coordenadas de $x$.
	\item A \textbf{distância} entre $x, y \in \mathbb{R}^n$ é dada por $||x-y||$.
\end{itemize}

\subsection{Topologia de $\mathbb{R}^n$}
A noção de distância em $\mathbb{R}^n$ permite-nos extender alguns conceitos a $\mathbb{R}^n$ que permitem definir a estrutura topológica deste congunto.\\
Começamos com um dos conceitos mais fundamentais: o de vizinhança. Definimos uma \textbf{bola} de raio $\epsilon \in \mathbb{R}^+$ e centro $a \in \mathbb{R}^n$ como o conjunto $B_\epsilon(a) = \{ x \in \mathbb{R}^n : ||x-a|| < \epsilon \}$. É evidente que o conceito de bola é uma extensão do conceito de vizinhançaem $\mathbb{R}^n$, que não passa de uma bola uni-dimensional.\\
Com este conceito, podemos então classificar conjuntos de forma semelhande ao feito em CDI1:
\begin{itemize}
\item $a$ é um \textbf{ponto interior} de $A$ se existe $r$ tal que $B_r(a) \subset A$ \vspace{-0.8mm}
\item $a$ é um \textbf{ponto exterior} de $A$ se existe $r$ tal que $B_r(a) \subset \mathbb{R}^n/A$ \vspace{-0.8mm}
\item $a$ é um \textbf{ponto fronteiriço} de $A$ se não é interior nem exterior a $A$ \vspace{-0.8mm}
\item $a$ é um \textbf{ponto aderente} a $A$ se para qualquer $r$, $B_r(a) \cap A \neq \emptyset$ \vspace{-0.8mm}
\end{itemize}
e chamamos: \vspace{-0.8mm}
\begin{itemize}
\item \textbf{Interior} de $A$ ($int \, A$) ao conjunto de pontos interiores de A \vspace{-0.8mm}
\item \textbf{Exterior} de $A$ ($ext \, A$) ao conjunto de pontos exteriores de A \vspace{-0.8mm}
\item \textbf{Fronteira} de $A$ ao ($\partial A$) conjunto de pontos fronteirições de A \vspace{-0.8mm}
\item \textbf{Fecho} ou \textbf{aderência} de $A$ ($\overline{A}$) ao conjunto de pontos aderentes de A ($\overline{A}=int \, A \cup \partial A$) \vspace{-0.8mm}
\end{itemize}
Observe-se que para qualquer $A \subset \mathbb{R}^n$, temos que 
$int \, A \cup \partial A \cup ext \, A = \mathbb{R}^n$ e $int \, A \cap \partial A \cap ext \, A = \emptyset$.\\
Um conjunto $A \in \mathbb{R}$ diz-se: \vspace{-0.8mm}
\begin{itemize}
\item \textbf{aberto} se $A= int \, A$ \vspace{-1mm}
\item \textbf{fechado} se $A= \overline{A}$ ($\mathbb{R}/A$ é aberto ou $\partial A \subset A$) \vspace{-0.8mm}
\item Um conjunto $A$ diz-se \textbf{limitado} se houver $x \in \mathbb{R}$ e $R \in \mathbb{R}^+$ tal que $A \subset B_R(x)$.
\item \textbf{compacto} se é fechado e limitado \vspace{-0.8mm}
\end{itemize}
A interseção/união de uma família de conjuntos abertos/fechados é também aberto/fechado.

\subsection{Sucessões}
Recorda-se que uma sucessão é essencialmente uma bijeção/correspondência entre um conjunto $\{ u_n: n \in \mathbb{N}^+\}$ e $\mathbb{N}^+$.
Seja $u_n$ uma sucessão de termos em $\mathbb{R}^n$. Observa-se que cada coordenada de $u_n$ descreve ela própria um sucessão de termos em $\mathbb{R}^n$. Assim, para cada $j \leq m$ damos o nome de \textbf{sucessões coordenadas} às sucessões $(u_{j_n})$ cujo termo de ordem $k$ corresponde à $j$-ésima coordenada de $u_k$.\\
Diz-se que uma sucessão $(u_n) \subset \mathbb{R}^n$ tende/converge para $u \in \mathbb{R}^n$ (e escreve-se $u_n \to u$) se e só se, para qualquer $\epsilon \in \mathbb{R}^+$ existe $p \in \mathbb{N}^+$ tal que $u_n \in B_\epsilon(u)$ para todo o $n>p$. É fácil de ver que esta definição é equivalente a verificar se a sucessão $d_n = || u_n - u ||$ é infinitesimal. Uma sucessão que convirja para $u \in \mathbb{R}^n$ diz-se \textbf{convergente}.\\
Note-se que as definições a cima enunciadas de convergência são pouco práticas. De facto, é bastante mais prático usar o seguinte facto:\\
Uma sucessão $(u_n)$ converge para $a = (a_1, a_2, \cdots , a_n) \in \mathbb{R}^n$ se e só se as sucessões coordenadas $u_{j_n}$ tendem para $a_j$ para todo o $j \in \{ 1,2, \cdots , n \}$. É então fácil verificar que o limite de uma sucessão em $\mathbb{R}^n$ é único (se existir).\\
Observe-se que qualquer sucessão de termos em $X \in \mathbb{R}^n$ tem limite na aderência de $X$. Então um conjunto é fechado se e só se qualquer sucessão de termos nesse conjunto tiver limite dentro do conjunto.\\
Diz-se ainda que uma sucessão é limitada se e só se o conjunto dos seus termos também o for. Tendo em conta os resultados a cima, é fácil concluir que uma sucessão é limitada se e só se as suas sucessões coordenadas também o forem. Verificam-se as seguintes propriedades:
\begin{itemize}
	\item Qualquer sucessão convergente é limitada.
	\item \textbf{Teorema de Bolzano-Weiestrass}: Qualquer sucessão limitada tem sub-sucessões convergentes.
\end{itemize}

\subsection{Continuidade}
Uma função $f: D \subset \mathbb{R}^n \to \mathbb{R}^m$ diz-se contínua num ponto $a \in \mathbb{R}^n$ se para qualquer bola de raio $\delta \in \mathbb{R}^+$ centrada em $f(a)$, existe uma bola de raio $\epsilon \in \mathbb{R}^+$ centrada em $a$ tal que 
$$x \in B_\epsilon(a) \, \cap \, D \Rightarrow f(x) \in B_\delta(f(a)) \quad (\forall \delta \in \mathbb{R}^+, \, \exists \epsilon \in \mathbb{R}^+ : ||x-a|| < \epsilon \vee x \in D \Rightarrow ||f(x)-f(a)|| < \delta)$$
Tal como em $\mathbb{R}$ (definição de Heine), podemos definir continuidade através de sucessões. Uma funçao $f : D \subset \mathbb{R}^n \to \mathbb{R}^m$ é contínua em $a \in D$ se e só se para qualquer sucessão $(u_n)$ de termos em $D$ convergente para $a$ se tem que $f(u_k) \to f(a)$.\\
Tal como nas sucessões, por vezes o estudo da continuidade de uma função torna-se mais simples se considerarmos as \textbf{funções coordenadas}. Seja uma função $f: D \subset \mathbb{R}^n \to \mathbb{R}^m$. Se considerarmos apenas o que $f$ faz à $k$-ésima coordenada dos objetos de $D$, obtemos uma função $f_k(x)$, de domínio em $\mathbb{R}$ a qual designamos de $k$-ésima função coordenada de $f$. Tal função $f$ será contínua num ponto $a \in D$ se e só se as $m$ funções coordenadas também o forem nas respetivas corrdenadas.\\
Propriedades de funções contínuas:
Se $f,g: \mathbb{R}^n \to \mathbb{R}^m$ forem contínuas em $a \in \mathbb{R}$ e $h: \mathbb{R}^m \to \mathbb{R}^l$ for contínua em $f(a) \in \mathbb{R}$, e $\alpha \in \mathbb{R}$ então:
\begin{itemize}
	\item $f \pm g$, $f \cdot g$, $||f||$, $\alpha f$, $f/g$ com $g(x) \neq 0$, são contínuas em $a$;
	\item $(h \circ f)$ é contínua em $a$;
\end{itemize}
Finalmente, introduzimos a noção de conexividade. Dois conjuntos $A,B \subset \mathbb{R}^n$ dizem-se \textbf{separados} se 
$$
A \cap \overline{B} = \overline{A} \cap B = \emptyset
$$
Um conjunto $C$ diz-se \textbf{conexo} se não houverem $A,B \subset C$ não-vazios tais que $A$ e $B$ são separados.\\
Teoremas sobre funções contínuas:
\begin{itemize}
	\item Uma função contínua transforma conjuntos compactos em conjuntos compactos;
	\item \textbf{Teorema de Weiestrass}: Uma função escalar contínua de domínio compacto em $\mathbb{R}^n$ tem máximo e mínimo;
	\item Uma função contínua transforma conjuntos conexos em conjuntos conexos;
	\item \textbf{Teorema do Valor Intermédio}: Seja $f : D \subset \mathbb{R}^n \to \mathbb{R}$ contínua de domínio conexo. Se $a,b \in f(D)$ com $b<a$ então $[a,b] \subset f(D)$.
\end{itemize}

\subsection{Limites}
Tal como em CDI-I, definimos o \textbf{prolongamento contínuo} de uma função $f: D \subset \mathbb{R}^n \to \mathbb{R}^m$ como a função contínua $\tilde{f}: D^* \subset \overline{D} \to \mathbb{R}^n$ tal que $\tilde{f}(x) = f(x)$ para $x \in D$ e $D^*$ corresponde ao conjunto dos pontos em $\overline{D}$ tais que $f$ é prolongável por continuidade nesses pontos.\\
Então, definimos o limite de $f$ em $a \in \mathbb{R}$ como o valor do prolongamento contínuo de $f$ nesse ponto.
$$
\lim_{x \to a} f(x) = \tilde{f}(x)
$$
O limite de $f$ só existe nos pontos prolongáveis por continuidade.\\
Equivalentemente temos que para uma função $f: D \subset \mathbb{R}^n \to \mathbb{R}^m$:
$$
\lim_{x \to a} f(x) = b
$$
se e só se
\begin{itemize}
	\item Para qualquer sucessão de termos em $D$ tal que $u_n \to a$ se verifica que $f(u_n) \to f(a)$;
	\item $\forall_{\delta \in \mathbb{R}^+}, \exists_{\epsilon \in \mathbb{R}^+}: x \in B_\epsilon(a) \cap D \Rightarrow f(x) \in B_\delta(b)$.
\end{itemize}
Note-se que para um limite existir num ponto $a$, este tem de ser igual em todas as curvas que passam por esse ponto. Nomeadamente, o limite de uma função em $a$ só pode existir se o limite existir e for igual segundo todas as retas por $a$ (note-se que o limite segundo uma reta é um limite em $\mathbb{R}$). A estes limites (segundo retas) dá-se o nome de \textbf{limites direcionais} segundo uma reta/vetor.
No entanto, e como é impossível verificar o limite de uma função segundo todas as curvas que passam por um ponto, é necessário um método para calcular um limite.\\
\textbf{Método das funções enquadradas}:\\
Sejam $f,g,h: \mathbb{R}^n \to \mathbb{R}^m$ definidas em $B_r(a)$ para $a \in \mathbb{R}^n, r \in \mathbb{R}^+$. Então:
$$
f(x) \geq g(x) \geq h(x) \quad \forall_{x \in B_r(a)} \Rightarrow \lim_{x \to a} f(x) \geq \lim_{x \to a} g(x) \geq \lim_{x \to a} h(x) 
$$
e nomeadamente, se $\lim_{x \to a} f(x) = \lim_{x \to a} h(x) = b$ temos que $\lim_{x \to a} g(x) = b$.

\subsection{Diferenciabilidade}
Vimos em CDI-I que, entre outras, uma possível definição de derivada de uma função $f$ num ponto $a$ ($f'(a)$) é tal que 
$$
f(x) = f(a) + f'(a)(x-a) + r(x)
$$
em que $r(x)$ é uma função resto tal que $\frac{r(x)}{x-a} \to 0$ quando $x \to a$.\\
Note-se que a função $g(x) = f(a) + f'(a)(x-a)$ é uma função afim. De facto, esta é a melhor aproximação afim da função $f$.\\ 
Definimos então a derivada de uma função $f$ num ponto $a$ como a transformação linear $Df(a)$ que melhor aproxima $f$ na vizinhança de $a$. Note-se que $f$ vai então ser diferenciável se e só se esta transformação linear existir. Nesse caso, temos que:
$$
f(x) = f(a) + Df(a)(x-a) + r(x)
$$
Note-se que isto significa que a função $g(x) = f(a) + Df(a)(x-a)$ será o hiper-plano tangente á função $f$ no ponto $a$ (nomeadamente, para funções de domínio em $\mathbb{R}$, o hiper-plano é uma reta - a reta tangente).\\
Como qualquer transformação linear, $Df(a)$ pode ser representado por uma matriz. Se $f: \mathbb{R}^n \to \mathbb{R}^m$, $Df(a)$ vai ser representado por uma matriz $m \times n$, a qual se dá o nome de \textbf{matriz Jacobiana}.\\
Para construir a matriz Jacobiana, vemos o que esta faz aos vetores canónicos. Considere-se então $e_k$, o $k$-ésimo vetor canónico de $\mathbb{R}^n$. Vemos que:
$$
f(a + te_k) = f(a) + Df(a)(te_k) + r(te_k) \Leftrightarrow
$$
$$
Df(a)e_k = \lim_{t \to 0} \frac{f(a + te_k)-f(a)}{t}
$$
De forma que a matriz Jacobiana será constituída por $n$ vetores coluna dados pelo limite a cima. A este limite dá-se o nome de \textbf{derivada parcial} de $f$ em $a$, em ordem à variável $x_k$. Observando que o limite em cima origina um vetor com $m$ coordenadas, correspondentes aos limites
$$
\lim_{t \to 0} \frac{f_j(a + te_k)-f_j(a)}{t}
$$
em que $f_j$ representa uma das $m$ funções coordenadas de $f$, obtemos então a matriz:\\
\begin{center}
$Df(a) = $
$\begin{bmatrix}
	\frac{\partial f_1}{\partial x_1}(a) & \frac{\partial f_1}{\partial x_2}(a) & \cdots & 	\frac{\partial f_1}{\partial x_n}(a) \\
	\\
	\frac{\partial f_2}{\partial x_1}(a) & \frac{\partial f_2}{\partial x_2}(a) & \cdots & 	\frac{\partial f_2}{\partial x_n}(a) \\
	\\
	\cdots & \cdots & \cdots & \cdots \\
	\\
	\frac{\partial f_m}{\partial x_1}(a) & \frac{\partial f_m}{\partial x_2}(a) & \cdots & 	\frac{\partial f_m}{\partial x_n}(a) \\
\end{bmatrix}$
\end{center}
A noção de derivada parcial pode ser generalizada para vetores que não os canónicos. Define-se \textbf{derivada direcional} segundo um vetor $v$ (caso exista) como:
$$
\frac{\partial f}{\partial v}(a) = \frac{\partial}{\partial t} f(a+tv) \Big|_{t=0} = \lim_{t \to 0} \frac{f(a+tv)-f(a)}{t}
$$
A derivada segundo qualquer vetor $v$ pode no entanto ser obtida por meio da matriz Jacobiana. Para uma função $f: D \subset \mathbb{R}^n \to \mathbb{R}^m$ diferenciável em $a$, temos que, para qualquer $v \in \mathbb{R}^n$:
$$
\frac{\partial f}{\partial v}(a) = Df(a) \cdot v
$$
($\frac{\partial f}{\partial v}(a)$ é a imagem de $v$ por $Df(a)$).

\end{document}
